\documentclass{article}

% Russian language support
\usepackage[russian]{babel}
\usepackage[utf8]{inputenc}
\usepackage{hyperref}
\usepackage{graphicx}

\begin{document}

\title{Micromouse docs}
\author{I am}
\date{\today}
\maketitle

\section{Введение}

Микромаус   ---     Гонки робомышей на время и тд и тп.

\section{Основная часть}


\newpage
\section{собственно --- \emph{правила}. }

\subsection*{Правила для роботов Micromouse}

\begin{enumerate}
    \item Роботы Micromouse должны быть самостоятельными и автономными. Использование источников энергии на основе горения запрещено.
    \item Запрещено добавлять, удалять, заменять или изменять какие-либо компоненты робота (включая батареи) во время соревнования. Незначительный ремонт и регулировка допускаются с разрешения судей. В случаях, которые судьи сочтут особо необходимыми, может быть разрешена замена батарей на аналогичные по характеристикам.
    \item Робот не должен оставлять в лабиринте никаких своих частей.
    \item Роботу запрещено перепрыгивать через стены лабиринта, взбираться на них, а также использовать методы Sensing, которые могут привести к повреждению лабиринта.
    \item Размеры робота должны в любой момент позволять разместить его в квадрате со стороной 25 см на поверхности пола. Ограничений по высоте нет.
\end{enumerate}

\subsection*{Правила, касающиеся лабиринта}

\begin{enumerate}
    \item Боковые стороны стен лабиринта — белые, верхушки стен — красные, пол — чёрный. Беговая поверхность лабиринта должна быть окрашена чёрной матовой краской или лаком.
    \item Лабиринт состоит из модульных квадратов размером 18 см x 18 см, а его общий размер составляет 16 x 16 квадратов. Высота стен — 5 см, толщина — 1,2 см. Проходы между стенами имеют ширину 16,8 см.
    \item Стартовый квадрат расположен в одном из четырёх углов и ориентирован по часовой стрелке. Стартовый квадрат имеет стены с трёх сторон. Ориентация стартового квадрата такова, что когда открытая стена находится на «севере», внешние стены лабиринта расположены на «западе» и «юге». Цель представляет собой прямоугольную область, охватывающую несколько квадратов. Положение и размер зоны финиша определяются заранее для каждого соревнования. Зона финиша описывается координатами двух диагонально противоположных квадратов. Для классического состязания зоной финиша обычно являются четыре центральных квадрата лабиринта.
    \item Квадратные столбики размером 1,2 см x 1,2 см x 5 см высотой размещаются в четырёх углах каждого модульного квадрата (в узлах сетки). Лабиринт должен быть устроен таким образом, чтобы по крайней мере одна стена касалась каждого узла сетки, за исключением центра конечной точки. Кроме того, все периферийные стены всего лабиринта присутствуют.
\end{enumerate}

\subsection*{Правила проведения соревнований}

\begin{enumerate}
    \item Наименьшее время, затраченное роботом на прохождение от стартовой точки до финишной, принимается за время прохождения лабиринта.
    \item Оператор не должен вводить информацию о лабиринте в робота после того, как лабиринт был представлен. Кроме того, оператору запрещается вручную корректировать или частично стирать информацию о лабиринте во время соревнования.
    \item Забег по лабиринту всегда начинается со стартового квадрата и заканчивается, когда робот возвращается в стартовый квадрат, останавливается на 2 секунды или более, или санкционирована ручная эвакуация робота.
    \item Если робот возвращается в стартовый квадрат и перезапускается автоматически, он должен остановиться в стартовом квадрате как минимум на 2 секунды.
    \item Оператор не должен прикасаться к работающему роботу, если только судьи не дали указания или разрешения на это. Судья соревнований разрешит ручную эвакуацию, если робот явно неисправен или не может продолжать движение. Кроме того, запрос на эвакуацию при других обстоятельствах может быть принят при условии, что вся память, касающаяся лабиринта, будет стёрта.
    \item Роботам предоставляется 10 минут, в течение которых они могут совершить до 5 заездов. Однако на соревнованиях, где судьи сочтут это необходимым, время может быть сокращено до 7 минут или менее для размещения всех участников в отведённое время конкурса.
    \item Считается, что робот прошёл через временные ворота, когда все части робота в пределах 5 см от поверхности пола вошли в зону финиша. Однако измерение времени забега производится, когда датчик в стартовой точке фиксирует робота, а затем датчик в конечной точке фиксирует того же робота.
    \item Освещение, температура и влажность на соревновании должны соответствовать обычной внутренней среде. Запросы на корректировку освещения приниматься не будут.
    \item Судьи могут попросить оператора объяснить принцип работы робота, если сочтут это необходимым. На усмотрение судей участникам может быть предложено прекратить выступление, они могут быть дисквалифицированы или к ним могут быть применены иные необходимые меры.
    \item Содержание и критерии награждения соревнования определяются для каждого конкретного соревнования.
\end{enumerate}

\subsection*{Примечания}

\begin{itemize}
    \item Во время соревнования запрещена загрузка программ и смена ПЗУ. Кроме того, во время соревнования запрещено подключать робота к внешнему устройству разработки или консоли, не входящей в основное устройство, с целью дачи инструкций по выполнению программы.
    \item Когда робот завершил заезд и остановился как минимум на 2 секунды, его можно извлечь из лабиринта для мелкого технического обслуживания, такого как очистка шин.
    \item Хотя во время соревнования разрешено удалять пыль и грязь с шин с помощью клейкой ленты, не следует использовать растворители и т.п.
    \item Робот может продолжать исследовать лабиринт после достижения финишной точки в каждом заезде. В этом случае фиксируется время от стартовой точки до финишной.
    \item Если робот перезапускается в течение 2 секунд после возвращения в стартовую точку, считается, что начался следующий заезд, но хронометраж этого заезда является недействительным.
    \item Для регулировки и т.п. handlers не должны размещать робота в любой области, кроме стартового квадрата лабиринта.
    \item Лабиринты изготавливаются с общепринятой точностью, поэтому возможны некоторые погрешности размеров. Кроме того, на стенах и полу могут быть зазоры или перепады высот до 1 мм. Кроме того, на полу или стенах могут присутствовать variations цвета, обесцвечивание, пятна и тому подобное.
    \item Временные датчики представляют собой фотоэлектрические датчики проходного типа. Оптическая ось горизонтальна и находится на высоте 1 см над полом. Временные датчики размещаются на границе стартового квадрата и на границе любых входов в зону финиша.
    \item В зоне финиша нет стен или столбов.
    \item Если робот был убран со стартового квадрата после того, как был размещён там для начала заезда, то этот заезд считается завершённым.
    \item Судьи оставляют за собой право вносить изменения в любые из вышеуказанных правил в интересах честной игры и спортивного мастерства, а также для обеспечения того, чтобы все участники получили удовольствие от соревнования. В случае неоднозначности толкования, интерпретация судей любых пунктов правил является преобладающей.
\end{itemize}

\newpage

Пример робомыши 
с сайта \url{https://globaltronic.pt/en/product/micromouse-kit/}

\begin{figure}[h!]
\centering
\includegraphics[width=0.8\textwidth]{image.png}
\caption{Компоненты робота}
\label{Компоненты робота}
\end{figure}

\begin{enumerate}
    \item Левый ИК-датчик
    \item Передний левый ИК-датчик  
    \item Отсек для AA батареек (4 шт.)
    \item Включение/выключение AA батареек
    \item Левый шаговый двигатель
    \item Конфигурируемые переключатели для ПО управления
    \item Разъём для правого двигателя
    \item Правый ИК-датчик
    \item Передний правый ИК-датчик
    \item Разъём Bluetooth (модуль не входит в комплект)
    \item Светодиод pin 13 / Зуммер (зуммер не входит в комплект)
    \item Светодиод включения
    \item Перемычка для выбора питания: AA батарейки/LiPo аккумулятор
    \item Правый шаговый двигатель
    \item Включение/выключение LiPo аккумулятора (аккумулятор не входит в комплект)
    \item Разъём для левого двигателя
\end{enumerate}



\section{Заключение}

Это заключительная часть документа. Русский язык работает правильно!

\section{Ресурсы}

Список ресурсов на которые можно обратить внимание:

\begin{enumerate}
    
    \item Минские робо приколы, имеется как лайн фолловер так и мышиные бега 
    \href{https://robofinist.ru/event/info/competitions/id/1415}{\raisebox{-0.2ex}{\tiny *Кликни на меня!*}}
    \item Симулятор мыши проходящей лабиринт
    \href{https://github.com/mackorone/mms}{\raisebox{-0.2ex}{\tiny *Кликни на меня!*}}
\end{enumerate}







\end{document}