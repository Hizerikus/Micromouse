\documentclass{article}

% Russian language support


\usepackage[utf8]{inputenc}

\usepackage{hyperref}
\usepackage{graphicx}


\begin{document}

\title{Micromouse docs}
\author{I am}
\date{\today}
\maketitle

\section{Введение}

Микромаус   ---     Гонки робомышей на время и тд и тп.

\section{Основная часть}


\newpage

Пример робомыши 
с сайта \url{https://globaltronic.pt/en/product/micromouse-kit/}

\begin{figure}[h!]
\centering
\includegraphics[width=0.8\textwidth]{image.png}
\caption{Компоненты робота}
\label{Компоненты робота}
\end{figure}

\begin{enumerate}
    \item Левый ИК-датчик
    \item Передний левый ИК-датчик  
    \item Отсек для AA батареек (4 шт.)
    \item Включение/выключение AA батареек
    \item Левый шаговый двигатель
    \item Конфигурируемые переключатели для ПО управления
    \item Разъём для правого двигателя
    \item Правый ИК-датчик
    \item Передний правый ИК-датчик
    \item Разъём Bluetooth (модуль не входит в комплект)
    \item Светодиод pin 13 / Зуммер (зуммер не входит в комплект)
    \item Светодиод включения
    \item Перемычка для выбора питания: AA батарейки/LiPo аккумулятор
    \item Правый шаговый двигатель
    \item Включение/выключение LiPo аккумулятора (аккумулятор не входит в комплект)
    \item Разъём для левого двигателя
\end{enumerate}


\section{собственно карта маршрута - }


\section{Заключение}

Это заключительная часть документа. Русский язык работает правильно!

\section{Ресурсы}

Список ресурсов на которые можно обратить внимание:

\begin{enumerate}
    
    \item Минские робо приколы, имеется как лайн фолловер так и мышиные бега 
    \href{https://robofinist.ru/event/info/competitions/id/1415}{\raisebox{-0.2ex}{\tiny *Кликни на меня!*}}
    \item Симулятор мыши проходящей лабиринт
    \href{https://github.com/mackorone/mms}{\raisebox{-0.2ex}{\tiny *Кликни на меня!*}}
\end{enumerate}







\end{document}