\documentclass[12pt,a4paper]{article}

% Package for font encoding
\usepackage[T2A]{fontenc}

% Package for input encoding (if needed for older systems)
\usepackage[utf8]{inputenc}

% Package for Russian language support
\usepackage[russian]{babel}

% Improved font rendering
\usepackage{mathtext}

% Extended font support
\usepackage{amsmath,amsfonts,amssymb}

% For proper quotes and other typographic features
\usepackage{csquotes}

% For custom margins
\usepackage[left=2.5cm,right=1.5cm,top=2cm,bottom=2cm]{geometry}

% For including graphics
\usepackage{graphicx}

% For hyperlinks in the document
\usepackage{hyperref}

% For referencing PDF docs?
\usepackage{pdfpages}

% for dirtree lol 
\usepackage{dirtree}

% for diagrams
\usepackage{tikz}
\usetikzlibrary{arrows.meta, shapes.geometric, positioning, fit,backgrounds}

% Color scheme
\definecolor{primary}{RGB}{44,62,80}    % Dark blue-gray
\definecolor{accent}{RGB}{52,152,219}   % Bright blue
\definecolor{secondary}{RGB}{149,165,166} % Gray
\definecolor{background}{RGB}{236,240,241} % Light gray



\begin{document}

% Title page
\begin{titlepage}
    \centering
    \vspace*{1cm}
    
    {\LARGE\textbf{Записи по разработке Загрузчика}}\\
    \vspace{1cm}
    
    {\large для AudioStim на STM32F4XX}\\
    \vspace{2cm}
    
    {\large\textbf{Автор:} Гаврилов Е.В.}\\
    \vspace{10.5cm}
    
    {\large Иваново, \the\year}
    
\end{titlepage}

% Table of contents
\tableofcontents
\newpage

% Main content
\section{Введение}

        Эти записи предназначены для личного пользования во время разработки 
    Загрузчика (Bootloader), с дальнейшим применением для напсания статей.


        Разработка была начата на основе уже существующего загруччика для:
    \emph{Screen\_STM32F103xx\_DFU} в качестве примера 
    и уже имеющимся КПО на базе QT и NeuroUSB
    
    
    Процесс разработки делится на две части:
    \begin{enumerate}
        \item Уже имеющееся КПО с возможностью добавления новых функций
        \item Новое решение на базе StmDFU 
    \end{enumerate}
\section{Встраиваемая часть}

\emph{Древо файлов проекта.}

\dirtree{%
    .1 AudioStim\_STM32F429xx\_DFU - корень проекта загрузчика.
    .2 Core.
    .3 Inc.
    .4 main.h - здесь находятся флаги.
    .3 Src.
    .4 main.c - здесь лежит 
    superloop проверяющий загруженное ПО по окончанию её записи во флеш.
    .2 Middlewares.
    .3 \dots
    .6 DFU.
    .7 Inc.
    .8 usbd\_dfu.h - здесь энумерация комманд загрузчика; 
    сюда же передаются указатели на функции работы 
    с флем памятью из usbd\_dfu\_if.c.
    .7 Src.
    .8 usbd\_dfu.c.
    .2 USB\_DEVICE.
    .3 App.
    .4 usbd\_dfu\_if.c - здесь методы работы с флеш памятью.
    .4 usbd\_dfu\_if.h.
    .4 \dots.
    .3 Target.
    .4 \dots.
    .2 STM32F429ZITX\_FLASH.ld - Здесь обозначается размер загрузчика и 
    соответственно та область что остаётся под основную программу.
}

\newpage
\section{КПО часть}

    \emph{последовательность комманд после окончания энумерации (функционал КПО - функция writeInAdress):}

    \begin{tikzpicture}[
    node distance=1cm and 2cm,
    box/.style={
        rectangle,
        rounded corners=3pt,
        minimum width=3cm,
        minimum height=1cm,
        align=center,
        draw=primary,
        fill=background,
        line width=0.8pt,
        font=\sffamily
    },
    decision/.style={
        diamond,
        aspect=1.5,
        minimum width=2cm,
        align=center,
        draw=primary,
        fill=background,
        line width=0.8pt,
        font=\sffamily
    },
    arrow/.style={
        -{Stealth[length=3mm, width=2mm]},
        color=primary,
        line width=0.8pt
    },
    label/.style={
        font=\sffamily\small,
        color=secondary
    }
    subgroup style/.style={
draw=accent, dashed, thick, inner sep=12pt,
    }
]

% Nodes
\node[box] (start) {КПО request};
\node[box, below=of start] (DFUsendLen) {Отправка длинны прошивки (DFU\_SendLen)};
\node[box, below=of DFUsendLen] (DFUSendChecksum) {Отправка чексуммы (DFU\_SendChecksum)};
\node[box, below=of DFUSendChecksum] (setAddressregular) {Отправка длинны прошивки (DFU\_SendLen)};


% \node[decision, below=of process] (check) {Quality Check};
% \node[box, right=of check] (analysis) {Statistical Analysis};
% \node[box, below=of check] (optimize) {Parameter Optimization};
% \node[box, below=of optimize] (output) {Output Results};

% Arrows
\draw[arrow] (start) -- (DFUsendLen); 
\draw[arrow] (DFUsendLen) -- (DFUSendChecksum); 

% \draw[arrow] (start) -- (process);
% \draw[arrow] (process) -- (check);
% \draw[arrow] (check) -- node[label, above] {Pass} (analysis);
% \draw[arrow] (check) -- node[label, right] {Fail} (optimize);
% \draw[arrow] (analysis) |- (output);
% \draw[arrow] (optimize) -- (output);

% % Feedback loop
% \draw[arrow, dashed, color=accent] (optimize) -- ++(-3,0) |- node[label, left, pos=0.25] {Re-evaluate} (process);

% % Title
% \node[above=0.5cm of start, font=\sffamily\Large\bfseries, color=primary] {Technical Process Flow};

\end{tikzpicture}

\newpage
% Bibliography (if needed)
\begin{thebibliography}{9}
\bibitem{example} 
Пример источника. 
Название статьи. 
Журнал, год.
\end{thebibliography}

\end{document}